\documentclass[preprint]{aastex}
%\usepackage{apjfonts}


%%%%%%%%%%%%%%%%%%%%%%%%%%%%%%%%%%%%%%%%%%%%%%%%%%%%%%%%%%%%%%%%%%%%%%%%%%%%%
\def\[#1]{\cite{#1}}
\def\eq#1{\begin{equation} #1 \end{equation}}
\def\eqarray#1{\begin{eqnarray} #1 \end{eqnarray}}
\def\non{\nonumber \\}

\def\Dnu     {\hbox{$\Delta\nu$}}
\def\E        {\hbox{${\cal E}$}}
\def\j        {\hbox{${\jmath}$}}
\def\Jbar     {\hbox{$\bar J$}}
\def\Ie       {\hbox{$I_e$}}
\def\about    {\hbox{$\sim$}}
\def\half     {\hbox{$\frac12$}}
\def\tot      {\tau_t}
\def\t(#1){\tau_{#1}}
\def\a(#1){\alpha_{#1}}
%%%%%%%%%%%%%%%%%%%%%%%%%%%%%%%%%%%%%%%%%%%%%%%%%%%%%%%%%%%%%%%%%%%%%%%%%%%%%

\begin{document}

\title
                 {MOLPOP Line Emission Output}

\author
{Moshe; May 12, 2017} % \today}

The entire line flux at position $\tau$ is
\eq{
               F(\tau) = \int_{-\infty}^{\infty}F_\nu(\tau)d\nu
                       = \Dnu\int_{-\infty}^{\infty}F_\nu(\tau)dx
}
We introduced the line cooling rate per unit area (erg cm$^{-2}$ s$^{-1}$) and
the cooling factor \j\ via
\eq{
                \Lambda = F(\tot) - F(0) = 4\pi\Dnu\,\j
}
taking account of emission from the two faces of the slab (eq. 4, 2006 CEP
paper). As we show in eq. 93 of our CEP notes (where we used \E\ for
$\Lambda$), in the case of 1--zone we have
\eq{
    \j = S\tot\beta(\tot) = \epsilon\ell\beta(\tot)
}
where $\epsilon$ is the emission coefficient and $\ell$ is the slab thickness.
Therefore, the $u \to l$ transition has
\eq{
   \Lambda_{u,l} = 4\pi\Dnu\epsilon_{u,l}\beta_{u,l}\ell
                 = h\nu_{u,l}A_{u,l}\beta_{u,l}n_u\ell
                 = h\nu_{u,l}A_{u,l}\beta_{u,l}\frac{n_u}{n}N
}
where $n_u$ is the population of the upper level, $n$ is the overall molecular
density and $N$ is its column density. Since $N$ is the overall number of
molecules per unit area, the cooling rate per molecule (erg s$^{-1}$ per
molecule) is
\eq{
   {\tt{cool}}_{u,l} = \frac{\Lambda_{u,l}}{N}
                     = h\nu_{u,l}A_{u,l}\beta_{u,l}\frac{n_u}{n}
}
and the flux is
\eq{
   F(\tot) = -F(0) = \half\Lambda_{u,l}
                   = \half{\tt{cool}}_{u,l} N
}

\section*{MOLPOP Implementation}

We use the variables \texttt{x(i)} so that
\eq{
  \tt n(i) = nmol*x(i)*we(i)
}
is the population of level $i$, with {\tt we(i)} its weight factor. Therefore
\eq{
   \tt n(i) A(i,j) = nmol*x(i)*[we(i) A(i,j)]
}
and in MOLPOP we absorb the weight factor into the $A$-coefficient. So after
that is done,
\eq{
   \tt cool(i,j) = ems(i,j)*esc(i,j)*x(i)
}
(in subroutine {\tt lines}) where
\eq{
   \tt ems(i,j) = h\nu(i,j)*A(i,j)
}
is defined right after $A$ is scaled with {\tt we(i)} at the end of subroutine
{\tt INPUT}. Therefore, the flux should be defined as
\eq{
   \tt F(i,j) = \half cool(i,j)*nmol*R
}
This will give the result in erg cm$^{-2}$ s$^{-1}$. Switch to MKS: W = 10$^7$
erg s$^{-1}$ and m$^2$ = 10$^4$ cm$^2$ so W m$^{-2}$ = 10$^{7 - 4}$ erg
cm$^{-2}$ s$^{-1}$ and erg cm$^{-2}$ s$^{-1}$ = 10$^{-3}$ W m$^{-2}$. Therefore

\eq{
 \tt F(i,j) = aux*cool(i,j) \qquad W\,m^{-2}
}
where
\eq{
   \tt aux = .5*10^{-3}\,nmol*R
}


\end{document}

\subsection{Line Emissivity}

Now we should be able to get the full line flux from $F = F(\tot) - F(0)$ where
\eq{
    F(\tot) = \DnuD\int F_\nu(\tot)dx
}
and a similar expression for $F(0)$. This produces pairs of the form
\eq{
    \int dx [E_3(\tau_1\Phi) - E_3(\tau_2\Phi)] =
    \int dx [\half - E_3(\tau_2\Phi)]
  - \int dx [\half - E_3(\tau_1\Phi)]
}
>From the results of problem 2.12 in the Maser book (pp 35--36)
\eq{
    \alpha(\tau) = \tau\beta(\tau) = \int dx [\half - E_3(\tau\Phi)]
}
giving us
\eq{
    \int dx [E_3(\tau_1\Phi) - E_3(\tau_2\Phi)] =
    \alpha(\tau_2) - \alpha(\tau_1)
}
Therefore,
\eq{
    F(\tot) = 2\pi\DnuD\sum_{i = 1}^z S_i
    \left(\alpha_{z,i-1} -\alpha_{z,i}\right),  \qquad
    -F(0) = 2\pi\DnuD\sum_{i = 1}^z S_i
    \left(\alpha_{i,0} - \alpha_{i-1,0}\right)
}
In the case of symmetric slabs it's enough to calculate just one of these and
multiply by 2 to get \E. In the general case,
\eq{\label{eq:E}
    \E = 2\pi\DnuD\sum_{i = 1}^z S_i
    \left(\alpha_{i,0} - \alpha_{i-1,0} - \alpha_{z,i} + \alpha_{z,i-1}\right)
}
This is a lot simpler than eq.\ \ref{eq:F1}, where we have
\eq{\label{eq:F2}
 \E = 4\pi\DnuD \j, \quad
 \j = \sum_{i = 1}^z S_ip_i\t(i,i-1)
                      = \sum_{i = 1}^z \Big[\a(i,i-1) S_i -
     \half\sum_{j \ne i}(\a(i,j) - \a(i-1,j) - \a(i,j-1) + \a(i-1,j-1))S_j\Big]
}
So we must verify this is the same result. To simplify handling the sums,
introduce a matrix
\eq{\label{eq:M}
 M_{i,j} = -\half(\a(i,j) - \a(i-1,j) - \a(i,j-1) + \a(i-1,j-1))
}
Because $\a(i,j) = \a(j,i)$ $M$ is a symmetric matrix and because $\alpha_{i,i}
= 0$ its diagonal elements are
\eq{
    M_{i,i} = \alpha_{i,i - 1}
}
The expression for \j\ in eq.\ \ref{eq:F2} is therefore
\eq{
 \j  = \sum_{i = 1}^z M_{i,i}S_i +
     \sum_{i = 1}^z\sum_{\stackrel{j = 1}{j \ne i}}^z M_{i,j}S_j
}
and so obviously
\eq{
    \j = \sum_{i = 1}^z\sum_{j = 1}^z M_{i,j}S_j
       = \sum_{j = 1}^z S_j \sum_{i = 1}^z M_{i,j}
}
Now
\eq{
   2\sum_{i = 1}^z M_{i,j} =
        - \sum_{i = 1}^z \left(\a(i, j) - \a(i - 1, j)\right)
        + \sum_{i = 1}^z \left(\a(i, j - 1) - \a(i - 1, j - 1)\right)
}
Work on the 1st sum:
\eq{
  \sum_{i = 1}^z \left(\a(i, j) - \a(i - 1, j)\right) =
  \sum_{i = 1}^z \a(i, j)
  - \sum_{k = 0}^{z - 1} \a(k, j)
  = \a(z,j) - \a(0,j)
}
Similarly the 2nd:
\eq{
  \sum_{i = 1}^z \left(\a(i, j - 1) - \a(i - 1, j - 1)\right)
  = \a(z,j - 1) - \a(0,j - 1)
}
so that
\eq{
    \j = \half\sum_{j = 1}^z S_j (\a(0,j) - \a(z,j) - \a(0,j - 1) + \a(z,j - 1))
}
With $\E = 4\pi\DnuD\j$, this is exactly the same as eq.\ \ref{eq:E}.

Now it is also quite simple to show that for a symmetric slab, 1-- and 2--zone
partitions are identical. In the case of 1--zone we have $z = 1$ and
\eq{
    \j = \half S(\a(1,0) - \a(1,1) - \a(0,0) + \a(1,0))
       = \half S\cdot2\a(1,0) = S\tot\beta(\tot)
}
For a 2--zone partition the source function is the same in both and
\eq{
    \j = \half S(\a(1,0) - \a(2,1) - \a(0,0) + \a(2,0)
               + \a(2,0) - \a(2,2) - \a(1,0) + \a(2,1))
       = \half S\cdot2\a(2,0) = S\tot\beta(\tot)
}

\newpage
\subsection{Emerging Intensity}
The emissivity \j\ gives the line cooling. To get the observed radiation from a
slab need the intensity from equation \ref{eq:I_em}
\eq{
    I_\nu(\tot,\mu) = \int_0^{\tot} e^{-(\tot - t)\Phi/\mu}
                         S(t){\Phi\over\mu}dt
    = \sum_{i = 1}^z S_i
     \int\limits_{\tau_{i - 1}}^{\tau_i}e^{-(\tot - t)\Phi/\mu}{\Phi\over\mu} dt
      = \sum_{i = 1}^z S_i
   \int\limits_{(\tot - \tau_{i - 1})\Phi/\mu}
   ^{(\tot - \tau_i)\Phi/\mu}\!\!\!-e^{-u}du
}
So
\eq{
    I_\nu(\tot,\mu) = \sum_{i = 1}^z S_i
   \left(e^{-(\tot - \tau_i)\Phi/\mu} -
         e^{-(\tot - \tau_{i - 1})\Phi/\mu}\right)
         = \sum_{i = 1}^z S_i
     \left(e^{-\tau_{z,i}\Phi/\mu} -  e^{-\tau_{z,i - 1}\Phi/\mu}\right)
}
Angular integration properly reproduces the flux from equation \ref{eq:Ft}
because
\eq{
    \int_0^1 e^{-\tau\Phi/\mu}\mu d\mu = E_3(\tau\Phi)
}

\subsection{More on the Effectively Thin Emissivity}

For the effectively thin case we found (eq.\ \ref{eq:eff-thin})
\eq{
 {d\j\over d\tau} = {\epsilon\over 1 - \epsilon}B(T)
                  = {C_{21}\over A_{21}}{e^{E/kT} - 1\over e^{E/kT}}\times
                    {A_{21}\over B_{21}}{1\over e^{E/kT} - 1}
                  = {C_{21} e^{-E/kT}\over B_{21}}
                  = {g_1 C_{12} \over g_2 B_{21}}
}
We also have
\eq{
    d\tau = {1\over4\pi\DnuD}g_2B_{21}E_{21}(n_1 - n_2) d\ell
}
Therefore
\eq{
    \E = 4\pi\DnuD\j = E_{21}\int g_1 C_{12}\, (n_1 - n_2)\, d\ell
}

\section{June 6, 2005: DERIVATIVES}

Derivatives of $\beta$ and $\alpha = \tau\beta$ are most easily calculated from
the representation
\eq{
  \alpha(\tau) = \int_{-\infty}^{\infty} dx \{\half - E_3[\tau\Phi(x)]\}
}
This gives
\eq{
  {d\alpha\over d\tau} = \int_{-\infty}^{\infty} -\Phi E_3'(\tau\Phi)dx
                       = \int_{-\infty}^{\infty} \Phi E_2(\tau\Phi)dx
}
So we have
\eq{
    \alpha' = \int_{-\infty}^{\infty} \Phi(x) E_2[\tau\Phi(x)]dx, \qquad
    \beta' = {1\over\tau}\left(\alpha' - \beta\right)
}

\section{July 27, 2005: It's a LINEAR PROBLEM!!!}

{\bf In the CEP formulation, the 2-level problem in $\tau$-space becomes A SET
OF LINEAR EQUATIONS!!!} Start with equation \ref{eq:S-p}
\eq{
    S = (1 - \epsilon)(1 - p)S + \epsilon B(T)
}
Instead of manipulating it to solve for $S$ in terms of $p$ and $B$ as we did,
bring it to the form
\eq{
    S + \left({1\over\epsilon} - 1\right)pS = B
}
Now make the zone division to get the equation for zone $i$
\eq{
    S_i + \eta p_iS_i = B(T_i)
    \qquad \hbox{where}\
    \eta  = \left({1\over\epsilon} - 1\right) = {N'_{cr}\over N}
}
But
\eq{
    p_iS_i = \beta_i S_i + {1\over\t(i,i - 1)}
            \sum_{\stackrel{j = 1}{j \ne i}}^z M_{i,j}S_j
}
where the matrix elements $M_{ij}$ are combinations of $\a(i,j)$ (eq.\
\ref{eq:M}). So the set of equations for the unknown $S_i$ is
\eq{
      \left(1 + \eta\beta_i\right)S_i
    + {\eta\over\t(i,i - 1)}\sum_{\stackrel{j = 1}{j \ne i}}^z M_{i,j}S_j
    = B(T_i)
}
Voila, a set of linear equations!!! This was right in front of us the whole
time. In fact, it is even more straightforward to get this directly from the
standard equation for the source function (eq.\ \ref{eq:S}): divide into zones,
introduce $\Jbar_i$ as in eq.\ \ref{ave} and get this result without ever
mentioning $p$.

\section{The Divergent $\alpha(\tau)$}

For the asymptotic behavior of $\alpha$ use equation 88b in Capriotti and
recall that his definition of $\tau_0$ is line center optical depth to the
center of the slab (text before eq.\ 83). So the relation to our notation is
\eq{
    \tau = 2\tau_0\sqrt{\pi}
}
That means that the expression he gives for $\tau_0 > 2.5$ is
\eq{
    \alpha(\tau) \simeq \left(\ln\;\tau/\sqrt{\pi}\right)^{1/2}
                      + {0.25\over\left(\ln\;\tau/\sqrt{\pi}\right)^{1/2}}
                      + 0.14
}
and the asymptotic behavior when $\tau \to \infty$ is $\alpha \sim
\sqrt{\ln\tau}$. Then how come our equations do not diverge? The reason is that
the differences entering into $M_{i,j}$ end-up proportional to the second order
derivative $\alpha''$. To show that, start with
\eq{
    \t(i) = \t(i - 1) + \t(i,i-1), \qquad
    \t(j) = \t(j - 1) + \t(j,j-1)
}
and assume $i > j$. Then
\eqarray{
    \t(i,j)   &=& \t(i) - \t(j) = \t(i-1,j-1) + \t(i,i-1) - \t(j,j-1)  \cr
    \t(i-1,j) &=& \t(i-1,j-1) - \t(j,j-1)                            \cr
    \t(i,j-1) &=& \t(i-1,j-1) + \t(i,i-1)
}
and so
\eqarray{
    \a(i,j)   &=& \a(i-1,j-1) + \alpha'(\t(i,i-1) - \t(j,j-1))       \cr
    \a(i-1,j) &=& \a(i-1,j-1) - \alpha'\t(j,j-1)                     \cr
    \a(i,j-1) &=& \a(i-1,j-1) + \alpha'\t(i,i-1)
}
Therefore, to 1st order in the optical thicknesses $\t(i,i-1)$ of single zones
we have
\eq{
 \a(i,j) - \a(i-1,j) - \a(i,j-1) + \a(i-1,j-1) = 0
}

%\section{July 29, 2005: Rate Equations Linear in the Source Function (AAR)}
%
%We start with the statistical equilibrium equations written in the CEP manner:
%\eqarray{\label{eq:final}
%    {dn_k^i\over dt} = -\sum_{l = 1}^{k - 1} A_{kl}p_{kl}^i n_u^i +
%      C_{kl}^i\left(n_k^i - n_l^i e^{-E_{kl}/kT^i}\right) \qquad      \\
%  \hskip0.2in   +\sum^{L}_{u = k + 1} {g_u\over g_k}
%       \left[A_{uk}p_{uk}^i n_u^i +
%      C_{uk}^i\left(n_u^i - n_k^i e^{-E_{uk}/kT^i}\right)\right]   \nonumber
%}
%where
%\eqarray{\label{eq:p_ul}
%     p^i_{ul} &= &\beta^i_{ul} - {1\over2\t(i,i-1)}
%     \sum_{\stackrel{j = 1}{j \ne i}}^z
%     {n_u^j\over n_u^i}\,{n_l^i - n_u^i\over n_l^j - n_u^j} \times \non
%      &&(\a(i,j) - \a(i-1,j) - \a(i,j-1) + \a(i-1,j-1)).
%}
%
%These equations can be rewritten to incorporate the source function and the optical depth, so that:
%\eqarray{\label{eq:final}
%    {dn_k^i\over dt} = -\sum_{l = 1}^{k - 1}  {4\pi\Delta\nu_D\over g_k E_{kl} l^i}
%    \left[ \tau_{kl}^{i,i-1} \beta_{kl}^i S_{kl}^i - \sum_{\stackrel{j = 1}{j \ne i}}^z M_{i,j} S_{kl}^j +
%    \tau_{kl}^{i,i-1} {(S_{kl}^i-B^i)\over \eta_{kl}^i} \right] \qquad      \\
%  \hskip0.2in   +\sum^{L}_{u = k + 1} {4\pi\Delta\nu_D\over g_u E_{uk} l^i}
%    \left[ \tau_{uk}^{i,i-1} \beta_{uk}^i S_{uk}^i - \sum_{\stackrel{j = 1}{j \ne i}}^z M_{i,j} S_{uk}^j +
%    \tau_{uk}^{i,i-1} {(S_{uk}^i-B^i)\over \eta_{uk}^i} \right],   \nonumber
%}
%
%which immediately shows that the statistical equilibrium equations are linear in the source functions. In fact, this is a non-linear
%equation because the optical depths non-linearly depend on the source functions. However, they can be obtained from the level populations,
%which are obtained from the solution of the linear systems
%\eq{
%B_{ul} (n_l^i-n_u^i) S_{ul}^i - A_{ul} n_u^i = 0 \qquad \forall i
%}
%for all the transitions between and upper level $u$ and a lower level $l$.
%%The population of the upper levels, they can be written as:
%%\eq{
%%n_u^i = {4\pi\Delta\nu_D\over A_{ul} g_u E_{ul} l^i} \tau_{ul}^{i,i-1} S_{ul}^i
%%}
%To these equations, we have to add the closure relation
%\eq{
%n^i = \sum_{k=1}^L g_k n_k^i
%}

\section{November 9, 2005: Other CEP Implementations (ME)}

The CEP fundamental point is that \Jbar\ is calculated directly from the formal
solution of radiative transfer without solving the differential equation. Our
approach of zone division and averaging is just one implementation. As noted by
Jacques Le Bourlot, we could implement CEP also on a grid. Indeed, that's
straightforward. Equation \ref{Jbar} is simply the standard result
\eq{\label{eq:Jbar2}
 \Jbar(\tau) = \half \int_0^{\tot} S(t)dt
           \int_{-\infty}^{\infty}\Phi^2 E_1(|t - \tau|\Phi) dx
}
Or, using the notation of equation 2.9 in Avrett \& Hummer (MNRAS, 130, 295,
1965),
\eq{
    \Jbar = \int_0^{\tot} S(t) K_1(|t - \tau|)dt
}
where
\eq{
    K_1(\tau) = \half \int_{-\infty}^{\infty}\Phi^2 E_1(\tau\Phi)dx
}
They analyze this function in their Appendix I. So for $p$ we simply have
\eq{
    p(\tau) = 1 - {1\over S(\tau)}\int_0^{\tot} S(t) K_1(|t - \tau|)dt
}
and we can proceed to the rate equation with discretization on a grid instead
of in zones. In fact, we can even employ the standard ALI-SCP method, with the
only difference that \Jbar\ is taken from eq.\ \ref{eq:Jbar2} rather than from
an integration that follows the solution of the radiative transfer equation. It
seems like such an approach would have all the benefits of ALI-SCP with the
added advantage of much simpler and accurate calculation of \Jbar\ that does
not suffer from the errors in determining $I_{\nu}(\mu,\tau)$ in each step. Not
clear why this has not been done until now (or hasn't it?).

\subsection{November 14, 2005 (AAR)}


In case it has not been done before, let us assume, as a first step, that the source function varies linearly between
two consecutive points in the grid. Let the grid be labeled with the value of $\tau_i$, so that $i=0,\ldots,N$. Since the source function
varies linearly between points $\tau_i$ and $\tau_{i+1}$, we can write its functional form for each interval as:
\eq{\label{eq:source_linear}
S(t) = \cases{a_i+b_i t   & for $\tau_i \leq t \leq \tau_{i+1}$   \cr
                       0 & elsewhere  \cr},
}
where
\eqarray{\label{eq:source_linear2}
a_i &=& S_i - \tau_i \frac{S_{i+1}-S_i}{\tau_{i+1}-\tau_i} \\
b_i &=& \frac{S_{i+1}-S_i}{\tau_{i+1}-\tau_i}
}
Eq. \ref{eq:Jbar2} can then be split up:
\eq{\label{eq:Jbar_split}
 \Jbar(\tau) = \half \int_0^{\tau} S(t)dt
           \int_{-\infty}^{\infty}\Phi^2 E_1\left[ (\tau - t)\Phi \right] dx + \half \int_\tau^{\tot} S(t)dt
           \int_{-\infty}^{\infty}\Phi^2 E_1 \left[ (t-\tau)\Phi \right] dx.
}
The first integral takes into account the contribution to the mean intensity of points between the surface and $\tau$, while the second takes into
account the contribution of points between $\tau$ and the bottom of the slab (given by $\tot$). We call the first contribution
$\Jbar_\mathrm{outer}(\tau_j)$ while the second is called $\Jbar_\mathrm{inner}(\tau_j)$. Obviously,
$\Jbar=\Jbar_\mathrm{outer}+\Jbar_\mathrm{inner}$. Interchanging the order of integration in eq. \ref{eq:Jbar_split}, and explicitly separating each integral for each interval $[\tau_i,\tau_{i+1}]$, and carrying out the analytical integrations, we end up with the following two contributions:
\eq{\label{eq:Jbar_in}
 \Jbar_\mathrm{outer}(\tau_j) = \half \sum_{i=0}^{j-1}
           \int_{-\infty}^{\infty}dx \Phi \Bigg[ (a_i+b_i \tau_j) E_2(p) + \frac{b_i}{\Phi}  \left( E_3(p)-e^{-p} \right) \Bigg]_{(\tau_j-\tau_i)\Phi}^{(\tau_j-\tau_{i+1})\Phi}
}
\eq{\label{eq:Jbar_out}
 \Jbar_\mathrm{inner}(\tau_j) = \half \sum_{i=j}^{n-1}
           \int_{-\infty}^{\infty}dx \Phi \Bigg[ -(a_i+b_i \tau_j) E_2(p) + \frac{b_i}{\Phi} \left( E_3(p)-e^{-p} \right) \Bigg]_{(\tau_i-\tau_j)\Phi}^{(\tau_{i+1}-\tau_j)\Phi},
}

If the variation of the source function is assumed to be parabolic, the previous expressions are modified to give:
\eq{\label{eq:Jbar_in}
 \Jbar_\mathrm{outer}(\tau_j) = \half \sum_{i=0}^{j-1}
           \int_{-\infty}^{\infty}dx \Phi \Bigg[
              (a_i+b_i \tau_j + c_i \tau_j^2) E_2(p) +
              \frac{b_i+2c_i \tau_j}{\Phi} \left( E_3(p)-e^{-p} \right) +
              \frac{c_i}{\Phi^2} \left( E_4(p)+pe^{-p} \right) \Bigg]_{(\tau_j-\tau_i)\Phi}^{(\tau_j-\tau_{i+1})\Phi}
}
\eq{\label{eq:Jbar_out}
 \Jbar_\mathrm{inner}(\tau_j) = \half \sum_{i=j}^{n-1}
           \int_{-\infty}^{\infty}dx \Phi \Bigg[ -(a_i+b_i \tau_j + c_i \tau_j^2) E_2(p) +
              \frac{b_i+2c_i \tau_j}{\Phi} \left( E_3(p)-e^{-p} \right) -
              \frac{c_i}{\Phi^2} \left( E_4(p)+pe^{-p} \right) \Bigg]_{(\tau_i-\tau_j)\Phi}^{(\tau_{i+1}-\tau_j)\Phi},
}


\section{Variable Linewidth; February 11, 2007}

Consider the variation of \DnuD\ through the source. Refer to the paper's
equations by number. The radiative transfer eq.\ 1 does not contain any
information about the linewidth. The frequency dependence enters only via the
line profile $\Phi(x)$, which is a normalized function of $x$, not of $\nu$.
Calculating \Jbar\ involves an {\em average} over frequency, so again this is
done in terms of $x$, not $\nu$, so nothing at all changes in the CEP level
population equations. The only quantity that knows anything about the linewidth
is the optical depth in terms of the level populations, given in eq.\ 32. So
here's the only modification to the formalism:
\eq{
    \hbox{paper's eq. 32:} \qquad  \DnuD \Rightarrow \Delta\nu_{ul}^{i}
}
where $\Delta\nu_{ul}^{i}$ is the Doppler width of the $u \to l$ line in the
$i$-th zone. Note that the lack of $ul$ subscript in the paper's equation was a
mistake. {\bf This is the only effect on the calculation of the level
populations}.  The reason is that the functional form of $\Phi$ does not
change; it is the same function of $x$ everywhere, only the correspondence
between $x$ and $\nu$ varies with position.

Now that we have the level populations, to calculate emerging intensity from
the top face of the slab we'd use eq.\ (69) above (page 12). On the left we
have $I_\nu(\tot,\mu)$. On the right we have the profile $\Phi(x)$ in an
integral over $t$. Now we have to take into account that {\bf the same $\nu$
implies a different $x$ in every zone}. Breaking the integral into a sum over
zones makes handling this straightforward. In eq.\ (72), $\Phi$ would be
evaluated at a different argument in each zone, so here's the modified form:
\eq{\label{eq:Ft2}
    F_\nu(\tot)  = 2\pi \sum_{i = 1}^z
    \left[E_3(\tau_{z,i}\Phi_i) - E_3(\tau_{z,i - 1}\Phi_{i})\right] S_i
    \qquad \hbox{where} \quad
    \Phi_i = \Phi\left({\nu - \nu_0\over\Delta\nu^{i}}\right)
}
This expression is correct even though it contains optical depth between the
zone and the surface. The reason is that it comes from the integration over a
single zone, where $\Phi = \Phi_i$. So the final expression for the line
emission is the same as it was in eq.\ (77), only $\Phi$ is evaluated at a
different argument in each zone:
\eq{
    F_\nu = F_\nu(\tot) - F_\nu(0) = 2\pi \sum_{i = 1}^z S_i
    \left[E_3(\tau_{z,i}\Phi_i)
        - E_3(\tau_i\Phi_i)
        + E_3(\tau_{i - 1}\Phi_i)
        - E_3(\tau_{z,i - 1}\Phi_i)
    \right]
}
The $\nu$-integration over $d\nu$ in zone $i$ becomes simply integral with
$\Delta\nu^{i}dx$, so in the final eq.\ (83) the only thing is to bring the
linewidth inside the summation, with a different value in each zone:
\eq{\label{eq:E2}
    \E = 2\pi\sum_{i = 1}^z S_i\Delta\nu^{i}
    \left(\alpha_{i,0} - \alpha_{i-1,0} - \alpha_{z,i} + \alpha_{z,i-1}\right)
}



\end{document}
