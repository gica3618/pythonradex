\documentclass[preprint,12pt]{aastex}
\usepackage{amssymb,amsmath}

 \textwidth=6.5in
 \textheight=9in
 \topmargin=-0.75in
 \headheight=.15in
 \headsep=.35in
 \oddsidemargin=0in \evensidemargin=0in
 \parindent=1.2em
 \parskip=0.5ex


%%%%%%%%%%%%%%%%%%%%%%%%%%%%%%%%%%%%%%%%%%%%%%%%%%%%%%%%%%%%%%%%%%%%%%%%%%%%%
\def\eq#1{\begin{equation} #1 \end{equation}}

\def\Dv       {\hbox{$\Delta v$}}
\def\Tbr      {\hbox{$T_{\rm br}$}}
\def\Tx       {\hbox{$T_{\rm ex}$}}
\def\TB       {\hbox{$T_{\rm B}$}}

%%%%%%%%%%%%%%%%%%%%%%%%%%%%%%%%%%%%%%%%%%%%%%%%%%%%%%%%%%%%%%%%%%%%%%%%%%%%%

\begin{document}

%\title
\thispagestyle{empty}

\section*{Integrated Brightness Temperature for MOLPOP-CEP\\
  Moshe, June 6, 2014}


For uniform conditions, the line brightness temperature at frequency shift $x =
v/\Dv$ from line center, where \Dv\ is the thermal linewidth, is
\eq{
   \Tbr(x) = \Tx\left(1 - e^{-\tau(x)}\right)
}
where \Tx\ is the line excitation temperature. The optical depth $\tau(x)$ is
\eq{
   \tau(x) = \frac{\tau_0}{\mu}e^{-x^2}
}
where $\tau_0$ is the optical depth at line center and $\mu$ the viewing angle
from the slab normal. Therefore, the line-integrated brightness in
K\,km\,s$^{-1}$ is
\eq{
  \TB \equiv \int \Tbr dv = \Tx\Dv\int \left(1 - e^{-\tau(x)}\right)dx
}
Since MOLPOP produces the linewidth \Dv\ in the quantity {\tt vt} (it is
internally converted to cm\,s$^{-1}$ and must be divided by 10$^5$) and all
internal optical depths are at line-center, implementing this expression is
straightforward. In CEP runs, the integration is performed zone by zone, then
added up.


\end{document}
