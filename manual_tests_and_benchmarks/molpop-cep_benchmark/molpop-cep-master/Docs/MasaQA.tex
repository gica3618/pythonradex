\documentclass[12pt]{article}
\usepackage{natbib}
%\usepackage[hypertex]{hyperref}
\usepackage{hyperref}
\hypersetup{
  pdftitle={MOLPOP-CEP},
  pdfauthor={Asensio Ramos and Elitzur},
  pdfpagemode=None,        % don't open with bookmarks panel
  colorlinks=true,         % false: boxed links; true: colored links
%  linkcolor= black,        % internal links
%  citecolor= black,        % links to bibliography
%  urlcolor=  black,
  linkcolor= blue,        % internal links
  citecolor= blue,        % links to bibliography
  urlcolor=  blue,
  breaklinks=true          % allow links to break over lines
}


 \oddsidemargin 0.0in
 \evensidemargin 0.0in
 \textwidth 6.5in
 \topmargin -0.75in
 \textheight 9.5in
 \pagestyle{empty}

%%%%%%%%%%%%%%%%%%%%%%%%%%%%%%%%%%%%%%%%%%%%%%%%%%%%%%%%%%%%%%%%%%%
\def\eq#1{\begin{equation} {#1} \end{equation}}

\def\separation {0.5cm}
\def\non{\nonumber \\}
\def\DnuD     {\hbox{$\Delta\nu_D$}}
\def\Jbar     {\hbox{$\bar J$}}
\def\j        {\hbox{$\jmath$}}
\def\N        {\hbox{$\cal N$}}
\def\Ie       {\hbox{$I_e$}}
\def\Ji       {\hbox{$\bar J^i_\mathrm{ext}$}}
\def\about    {\hbox{$\sim$}}
\def\x        {\hbox{$\times$}}
\def\half     {\hbox{$1\over2$}}
\def\Ncr      {\hbox{$N'_{\rm cr}$}}
\def\mic      {\hbox{$\mu$m}}
\def\ion#1#2  {#1\,{\small {#2}} }
\def\tot      {\tau_t}
\def\t(#1){\tau^{#1}}
\def\a(#1){\alpha^{#1}}
\def\M{MOLPOP-CEP}
\def\LVG      {\texttt{LVG}}
\def\slab     {\texttt{slab}}



\def\Dv       {\hbox{$\Delta v$}}
\def\Tbr      {\hbox{$T_{\rm br}$}}
\def\Tx       {\hbox{$T_{\rm ex}$}}
\def\TB       {\hbox{$T_{\rm B}$}}

%%%%%%%%%%%%%%%%%%%%%%%%%%%%%%%%%%%%%%%%%%%%%%%%%%%%%%%%%%%%%%%%%%%




\begin{document}

\title                  {\sc Masa's Queries}

\author{ Moshe; \today}
\date{}
\maketitle

Answers to Masa's queries from March 29, 2017. They should find their way to
the \M\ manual.
\begin{quote}
  \begin{verbatim}
1. At the end of the *.out file, I see int(Tb dv) [K km/s].
Is this the value for molecular line flux at each J-transition?
I have used this value for my scientific interpretation.
Is this correct?

In the middle of this file, I also see emission [erg/s/mol].
Is this also THE molecular line flux value at each J-transition?

I am confused about their relation. For reference, I attach
"HCN_LVGtest.inp" and "HCN_LVGtest.out". For simplicity,
IR pumping is not included in these files. For example,
for N(mol) = 3.33E+12,
J=1-0 : 6.09E-01 [K km/s]
J=2-1 : 6.65E-01 [K km/s]
J=3-2 : 1.95E-01 [K km/s].

The J=2-1 to J=1-0 flux ratio is 1.09, and J=3-2 to J=1-0 flux ratio
is 0.32 in [K km/s].

On the other hand, if I use [erg/s/mol] in the middle, for N(mol) = 3.33E+12,
J=1-0 : 5.16E-21 [erg/s/mol]
J=2-1 : 1.95E-20 [erg/s/mol]
J=3-2 : 1.53E-20 [erg/s/mol]

The J=2-1 to J=1-0 flux ratio is 3.78, and J=3-2 to J=1-0 flux ratio
is 2.97 in [erg/s/mol].

When we convert from [K km/s] to [erg/s], we need to multiply with
nu^3 (nu is frequency). So when the J=2-1 to J=1-0 flux ratio in [K km/s]
is 1.09, the J=2-1 to J=1-0 flux ratio in [erg/s] should be 1.09*8=8.72.
When the J=3-2 to J=1-0 flux ratio in [K km/s] is 0.32, the J=3-2 to J=1-0
flux ratio in [erg/s] should be 0.32*27=8.64. However, the ratios in [erg/s]
in the *.out file are 3.78 and 2.97 for "J=2-1 to J=1-0" and "J=3-2 to J=1-0",
respectively, which are different from the above simple relation.
Could you please give me advice? Do I misunderstand anything?

Which of [K km/s] or [erg/s/mol] should I use for molecular line flux comparison
with observations?

  \end{verbatim}
\end{quote}

The quantity int(Tb dv) [K km/s] is calculated as follows: For uniform
conditions, the line RJ-brightness temperature at frequency shift $x = v/\Dv$
from line center, where \Dv\ is the thermal linewidth, is found from
\eq{
    \Tbr(x) = \Tx\left(1 - e^{-\tau(x,\mu)}\right)
}
where \Tx\ is the line excitation temperature. The optical depth $\tau(x,\mu)$
is
\eq{
   \tau(x,\mu) = \frac{\tau_0}{\mu}e^{-x^2}
}
where $\tau_0$ is the optical depth at line center and $\mu$ the viewing angle
from the slab normal. Therefore, \M\ calculates the line-integrated brightness
in K\,km\,s$^{-1}$ from
\eq{
  \TB \equiv \int \Tbr dv = \Tx\Dv\int \left(1 - e^{-\tau(x)}\right)dx
}
In CEP runs, the integration is performed zone by zone, then added up.

The quantity emission [erg/s/mol] is calculated as follows: For a transition
between upper level $u$ and lower level $l$ with frequency $\nu_{ul}$,
$A$-coefficient $A_{ul}$ and escape probability $\beta_{ul}$, the emission in
erg/s/mol is calculated from
\eq{\label{eq:ems}
    ems(u,l) = h\nu_{ul}A_{ul}\beta_{ul}\frac{n_u}{n_{\rm mol}}
}
where $n_u$ is the number density of the upper level and $n_{\rm mol}$ is the
molecular density. This quantity is the same as the line cooling factor
$\jmath$ we introduced in Elitzur \& Asensio Ramos 2006 (see eqs. 5 and 21 in
that paper).

As you can see from the definitions, the quantities you consider are not
related to each other with a simple multiplicative factor and you cannot infer
their values for some transition from their values for another. Which one to
use for comparison with observations depends very much on the data at hand and
how it was obtained; that is why we list both, leaving the choice to the user.

\begin{quote}
  \begin{verbatim}
   2. At the end of *.out file, I also see Flu [Jy] and Io [W/m/Hz/st].
   I assume they are peak flux density. Since 1 [Jy] = 10^-26 [W/m/Hz],
   these values are virtually identical except for "st".  For example, for J=1-0,
   N(mol) = 3.33E+12 : 5.82E+09 [Jy], 3.01E-18 [W/m/Hz/st]: ratio=0.52E-27
   N(mol) = 5.93E+12 : 8.83E+09 [Jy], 4.86E-18 [W/m/Hz/st]: ratio=0.55E-27
   N(mol) = 1.05E+13 : 1.28E+10 [Jy], 7.54E-18 [W/m/Hz/st]: ratio=0.59E-27
   N(mol) = 3.33E+16 : 1.38E+11 [Jy], 1.10E-16 [W/m/Hz/st]: ratio=0.80E-27

   The ratios are different for different N(mol)? Why does this happen?
   Is the solid angle different, depending on N(mol)? Namely, do you calculate
   the size of line emitting region and is it different depending on N(mol)?
  \end{verbatim}
\end{quote}

$I_0$ is indeed the peak intensity but the line flux is actually the overall
line emission explained above (the quantity $ems$ in eq.\ \ref{eq:ems})
converted to Jy. We should indeed add all these definitions to the manual.

\begin{quote}
  \begin{verbatim}
   3. I used only LVG mode in MOLPOP, because I want to see the effect of
   IR pumping, when compared to RADEX calculations by collisional excitation.
   I refer to Elitzur & Asensio Ramos 2006 MNRAS 365 779.
   Is this sufficient, since I use only LVG mode? May I write the webpage of
   MOLPOP?
  \end{verbatim}
\end{quote}

Sure. If you want to compare to a RADEX LVG calculation you should use the LVG
mode in MOLPOP. 


\end{document}
